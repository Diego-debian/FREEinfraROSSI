
%\documentclass[12pt,twocolumn]{article}
\documentclass[12pt]{article}
\usepackage[spanish,english]{babel}
%\usepackage[spanish]{babel}
\usepackage[utf8]{inputenc}
\usepackage{graphicx}
\usepackage{epsfig}
\usepackage{multicol,caption}
\usepackage{amsthm} % Theorem Formatting
\usepackage{amssymb}    % Math symbols such as \mathbb
\usepackage{color}

\usepackage{hyperref}
\usepackage[none]{hyphenat} 

%\renewcommand{\tablename}{Tabla}
%\def\tablename{Cuadro}% por \def\tablename{Tabla}% 
\newenvironment{Figure}
{\par\medskip\noindent\minipage{\linewidth}}
{\endminipage\par\medskip}

\addto\captionsspanish{%
  \def\tablename{Tabla}%
}

\topmargin  = 10pt
\oddsidemargin  = -0.5in
%\headheight = 12pt
%\headsep    = 15pt
%\footskip   = 15pt
\textheight = 21.5 cm
\textwidth  = 18.5cm

\tolerance=10000

\title{\bf{Experimento demostrativo del periodo de oscilación del péndulo, para el aula de clase}}
\author{Julian Salamanca\footnote{jasalamanca@udistrital.edu.co}, Diego Parra\footnote{diegoestudianteud1@gmail.com} \\
  Universidad Distrital, Calle 3 No 26A-40 Bogotá-Colombia\\
  Grupo Física e Informática ``FISINFOR''
}
\date{\today}
\begin{document}


%\def\tablename{Cuadro}% por \def\tablename{Tabla}% 
\renewcommand{\tablename}{Tabla}
\maketitle
\vspace{-0.8cm}

\begin{abstract}
This article provides an educational tool for displaying the oscillation period of the simple pendulum and harmonic movement, an experimental arrangement of an infrared sensor type both sender and receiver, which is operated by a atmega microcontroller 328P-Pu illustrated, sends data through a module HC-06 Bluetooth to a computer with an operating system GNU-Linux family as it is Linux Mint 17 Xfce, which analyzes real-time graphical and we provides important data on the characterization of the system rope – mass, in a controlled infrared radiation and vibrations that can alter our system pendulum swing environment; besides demonstration, the whole project was done with free software and hardware, is an experience that generates classroom projects `` doing physical ''.\\

{\bf{Keywords:}} Pendulum and simple harmonic motion, physics education, free software and hardware..
\end{abstract}
\selectlanguage{spanish}
\begin{abstract}

El presente artículo aporta una herramienta didáctica para la visualización del periodo de oscilación del péndulo  simple y el movimiento armónico,  se ilustra una disposición experimental de un sensor de tipo infrarrojo  tanto emisor como receptor,  el cual es operado por un microcontrolador atmega 328P-Pu, envía datos a través  un modulo bluetooth hc-06 a un ordenador con un sistema operativo de la familia GNU-Linux como lo es Linux Mint 17 xfce, el cual los analiza, gráfica en tiempo real  y nos arroja  datos importantes sobre la caracterización del sistema cuerda – masa, en un ambiente controlado de radiación infrarroja y vibraciones que puedan alterar nuestro sistema de  oscilación del péndulo; además de demostrativo, todo el proyecto fue realizado con software y hardware libre, es una experiencia que permite generar proyectos de aula ``haciendo física''.\\

{\bf{Descriptores:}} Péndulo y movimiento armónico simple, enseñanza de la física, software y hardware libre.
\end{abstract}

%tabla de contenido sin numeracion
\renewcommand\contentsname{\centering TABLA DE CONTENIDO}
\thispagestyle{empty}
\setcounter{page}{1}
\tableofcontents
\clearpage

%lista de figuras
%\renewcommand\listfigurename{\centering LISTA DE FIGURAS}
%\listoffigures
%\clearpage

%lista de tablas
% \renewcommand\listtablename{\centering LISTA DE TABLAS}
% \listoftables
% \clearpage


\begin{multicols}{2}

\section{Introducción}
Los experimentos demostrativos son utilizados en sesiones magistrales de clase para visualizar y obtener una imagen de la experiencia física, cuyo objetivo primordial es, el hacer participación interactiva con el estudiante\cite{REDISH} en el estudio de los conceptos físicos involucrados. Sin embargo, hay una gran cantidad de conceptos en física que pueden ser ilustrados a través de una simulación o de un arreglo experimental sofisticado.\\ \\ Este artículo presenta una propuesta de experimento para el estudio del periodo de oscilación en un péndulo, el cual busca ser, además de demostrativo, una experiencia que permita generar proyectos de aula ``haciendo física''.

\section{Configuración experimental}
El primer hombre dedicado a la investigación\cite{GALILEO} y realizador de grandes descubrimientos entre ellos la explicación física del péndulo y considerado el primer observador de la ciencia fue el señor Galileo Galilei, el cual dijo acerca de este tópico que sin importar cuál es el espacio de oscilación de un péndulo, siempre tendrá el mismo ritmo. 
\\ \\
El arreglo experimental esta esquematizado en la Fig. 1 y constaba de una cuerda atada a un apoyo, una masa sujeta a la cuerda, y como en aquel tiempo los relojes eran muy escasos, el señor Galilei\cite{GALILEO} utilizo los pulsos que generaba los latidos de su corazón para medir el tiempo.

\begin{Figure}
\center
\includegraphics[width=4.0cm, height=4cm] {fig/fig1.png}
\captionof{figure}{Esquema ``Figura realizada en python-visual'' de la configuración experimental utilizada por el señor Galileo Galilei para medir el periodo de oscilación de un péndulo y su isocronismo.}
\label{fig:g1}
\end{Figure}

\subsection{Elaboración del instrumento de medición}

\subsubsection{Materiales}
Para la realización de este montaje se debe utilizar los siguientes componentes:
\begin{enumerate}
\item[a.] Un ordenador con sistema operativo GNU-Linux.
\item[b.] Microcontrolador atmega 328P-PU\footnote{ Arduino, pagina oficial, proyecto arduino. http://www.arduino.cc/} para realizar la parte de control del hardware.
\item[c.] Un diodo led infrarrojo receptor y emisor\footnote{ Infrared data sheet.  http://www.datasheetarchive.com/dlmain/Datasheets-31/DSA-617614.pdf}
\item[d.] Modulo bluetooth HC-06 para arduino, el cual permite la comunicación a distancia con el dispositivo, sin necesidad de un sistema físico cableado.
\item[e.] Transistor LM-7805CV\footnote{ Data sheet transistor LM-7805CV. http://pdf.datasheetcatalog.net/datasheet/fairchild/LM7805.pdf}, transformador de voltaje de 9 voltios a 5 voltios el cual alimenta el microcontrolador atmega328P-PU, y los demás dispositivos del proyecto.
\item[f.] Un cristal de 16 MHz.
\item[g.] Tarjeta arduino uno\footnote{ Arduino, pagina oficial, proyecto arduino. http://www.arduino.cc/}.
\item[h.] Tres resistencias de 500 ohms.
\item[i.] Una resistencia de 1 komhs.
\item[j.] Dos capacitores cerámicos de 12 picofaradios.
\item[k.] Dos capacitores de 10 microfaradios.
\item[l.] Una batería de 9 V.
\item[m.] Un pulsador pequeño.
\item[n.] Tres led de colores de 4mm. 
\end{enumerate}

\subsubsection{Montaje del hardware}
A continuación se explica paso a paso como armar el circuito eléctrico, el montaje se realiza en fritzing\footnote{Enlace a la pagina oficial del proyecto fritzing http://www.fritzing.org/home/}. Se explicara parte por parte del circuito y luego todas deben unirse en una sola.
\\ \\
Cabe mencionar que los esquemas de las figuras de la 1 a la 8 fueron tomadas de la pagina principal del proyecto arduino, y modificadas en fritzing; lo primero es preparar el microcontrolador para su primer uso, debido a que los microcontroladores atmega 328 P-PU no vienen listos para comenzar a programar; es necesario hacer un stand alone\footnote{Bootloader arduino https://arduinoelectronics.wordpress.com/2012/02/10/standalone-atmega-without-arduino-bootloader/} sobre el microcontrolador.
\\ \\ 
Se realiza el montaje del transformador de 9 voltios a 5 voltios, el cual se aprecia en la Fig. 2 y en la Fig. 3, esto se hace con dos capacitores electrolíticos de 10 microfaradios  y el transistor lm-7805cv\footnote{ Ver datasheet del transistor.}.
\\ \\
Ambos  capacitores deben ir en paralelo quedando su extremo negativo conectado a tierra y su extremo positivo uno a la entrada de 9 V y el otro a la salida de 5 V del transistor lm-7805cv, el transistor lm-7805cv consta de 3 pines uno de los cuales es la entrada de +9V, el pin de la mitad es tierra o GND y el pin extremo es la salida a 5V; tal como se muestra en la figura 2; en la figura 3, se le coloco un enchufe de salida de poder.

\begin{Figure}
\center
\includegraphics[width=7cm, height=4cm]{fig/esquematrans.png} 
\captionof{figure}{Esquema eléctrico de un transformador de 9V a 5V, utilizando un transistor lm 7805 cv. Los capacitores son electrolíticos de 10 microfaradios.}
\label{fig:g2}
\end{Figure}

\begin{Figure}
\center
\includegraphics[width=7.cm, height=4cm]{fig/montajetr5V.png}
\captionof{figure}{Montaje en protoboard del circuito regulador de la figura 2.}
\label{fig:g3}
\end{Figure}

El cristal oscilador de 16 mHz debe ir siempre conectado en el microcontrolador, esto  con el fin de ajustar los tres relojes internos que trae el integrado atmega 328 P-Pu, para esto se utiliza el cristal de 16 mHz junto con los dos condensadores cerámicos de 12 picofaradios, Fig. 4 y Fig. 5,  uno de los pines del cristal debe conectarse al pin 9 del microcontrolador y el otro extremo del cristal al pin 10; conectar un capacitor a cada extremo del cristal y los extremos de los capacitores a tierra, de tal forma que los capacitores quedan en paralelo. 
\\ \\
El botón es para reiniciar el microcontrolador en caso de necesitarlo, Fig. 4 y Fig. 5, colocando uno de los extremos del botón, al pin 1 del microcontrolador que es el pin de reset, el otro extremo del botón a una resistencia de 1 Komhs y el extremo de la resistencia a tierra.

\begin{Figure}
\center
\includegraphics[width=8cm, height=4cm]{fig/bluetoothesq.png} 
\captionof{figure}{Circuito eléctrico modulo bluetooth, conectado a un microcontrolador atmega 328 P-PU junto con un cristal, dos capacitores cerámicos, un botón de reset y una resistencia de 1 Komhs.}
\label{fig:g4}
\end{Figure}

El modulo bluetooth se debe conectar de la siguiente manera: el pin 2 de la tarjeta arduino es el Rx este va conectado al Tx del bluetooth, el pin 3 del microcontrolador es el Tx y va conectado al pin Rx del bluetooth, Fig. 4 y Fig. 5, el pin GND del bluetooth debe ir a tierra, ahora el pin de Vcc del bluetooth se conecta 5V. 

\begin{Figure}
\center
\includegraphics[width=7.cm, height=4cm]{fig/bluetoothmon.png}
\captionof{figure}{Montaje en protoboard del circuito regulador de la figura 4,  los cables negros son tierra, los cables rojos son voltaje, y los cables amarillos son conexiones.}
\label{fig:g5}
\end{Figure}

La parte negativa del diodo  emisor infrarrojo se conecta a un extremo de una resistencia de 500 ohms y el extremo de la resistencia a tierra, figura 6 y 7 ; la parte positiva del diodo emisor se conecta al pin 4 del microcontrolador atmega328, que es la salida digital No. 2. Ahora se conecta la parte positiva del diodo infrarrojo receptor a una resistencia de 1 Komhs y el otro extremo de la resistencia a tierra, se conecta el pin de tierra del diodo receptor infrarrojo a +5V, el pin 28 del microcontrolador conecta a la parte positiva del diodo infrarrojo. 

\begin{Figure}
\center
\includegraphics[width=7.cm, height=4cm]{fig/s_esquema.png}
\captionof{figure}{Conexiones de los sensores a la tarjeta micro-controladora atmega328, el diodo receptor infrarrojo es el que se encuentra en la parte izquierda de la figura. El diodo emisor infrarrojo es el que se encuentra en la parte derecha de la figura conectado con tierra a tráves de una resistencia, y, al pin 4 de su otro extremo con el microcontrolador.}
\label{fig:g6}
\end{Figure}

\begin{Figure}
\center
\includegraphics[width=7.cm, height=4cm]{fig/sensores.png}
\captionof{figure}{Montaje en protoboard del circuito los sensores de la figura 6,  los cables negros son tierra, los cables rojos son voltaje, y los cables amarillos son conexiones.}
\label{fig:g7}
\end{Figure}
\vspace{1cm}
El led rojo se conecta al pin  9 del microcontrolador, el led verde se conecta al pin 11, el led azul se conecta al pin 10, este montaje con el fin de identificar que proceso se esta realizando en el microcontrolador. Ahora se conectan todos los circuitos anteriormente descritos, en un solo circuito con la baquela para conexiones. figura 8 y 9.  

\begin{Figure}
\center
\includegraphics[width=7.cm, height=4cm]{fig/mon0.png}
\captionof{figure}{Montaje en baquela, en la parte superior los diodos tanto emisor como receptor, se colocaron elevados para que la oscilación del péndulo no golpee con ningún componente del montaje.}
\label{fig:g8}
\end{Figure}

\begin{Figure}
\center
\includegraphics[width=7.cm, height=4cm]{fig/mon1.png}
\captionof{figure}{Montaje final del dispositivo, en la parte derecha superior se coloca un soporte de silicona para los diodos infrarrojos, esto da estabilidad y firmeza, se enrolla con cinta aislante para evitar ruido por radiaciones no deseadas que lleguen transversalmente.}
\label{fig:g9}
\end{Figure}

\subsubsection{Instalación del software free pops}
Programa que  ilustra el periodo de oscilación de un péndulo  y calcula la longitud de la cuerda, la frecuencia de oscilación y la frecuencia angular.   Copyright (C) 2015-22-08  Universidad Distrital Francisco Jose, Grupo de física e informática, Dr Julian Andres Salamanca Bernal, Diego Alberto Parra Garzón.
\\ \\ 
El programa free pops es software libre; puedes redistribuirlo y / o modificarlo bajo los términos de la Licencia Pública General GNU publicada por la Fundación para el Software Libre; ya sea la versión 3 de la Licencia, o (a su elección) cualquier versión posterior. 
\\ \\ 
Este programa se distribuye con la esperanza de que sea útil, pero SIN NINGUNA GARANTÍA; ni siquiera la garantía implícita de COMERCIALIZACIÓN o IDONEIDAD PARA UN PROPÓSITO PARTICULAR. Vea la Licencia Pública General GNU para más detalles. 
\\ \\
Para la instalación de free pops es necesario hacer lo siguiente:
\begin{enumerate}
\item[a. ] Abrir una terminal de linux, una vez hecho esto se instala el programa git de la siguiente manera: sudo apt-get install git 
\item[b. ] Descargar el programa free pops de los repositorios de github escribiendo en la terminal: \\git clone https://github.com/Diego-debian/F\\REE\_POPS\_1.0.git 
\item[c. ] Una vez descargado free pops, escribir en la terminal:\\sudo python FREE\_POPS\_1.0/Install/instala\\dor.py 
\item[d. ] Ya abierto el instalador de free pops, pregunta si desea continuar con la instalación, escribir 1 y oprimir enter.
\item[e. ] Se despliega un sub-menu que pregunta si desea instalar o desinstalar free pops, escribir 1 y oprimir enter.
\item[f. ] El programa actualiza el sistema y pregunta si desea instalar las dependencias de free pops, escribir la tecla S y oprimir enter. 
\end{enumerate}
Una vez hecho esto ya esta instalado el programa. 

\begin{Figure}
\center
\includegraphics[width=8.cm, height=4cm]{fig/Docu1.png}
\captionof{figure}{Ventana de inicio del software free pops.}
\label{fig:g10}
\end{Figure}

\subsubsection{Primer uso de free pops}
Para el primer uso de free pops es necesario:
\begin{enumerate}
\item[a. ] Abrir el gestor de archivos.
\item[b. ] Entrar en la carpeta FREE\_POPS\_1.0.
\item[c. ] Entrar en la carpeta free\_pops.
\item[d. ] Hacer doble click izquierdo sobre el programa free\_pops y escribir la cable de superusuario.
\item[e. ] Conectar\footnote{ Ver figura 10.} la tarjeta arduino uno con el microcontrolador atmega328P-PU, a través del cable usb al computador y oprimir el botón firmware, se abre una ventana de dialogo, oprimir el botón ok. 
\item[f. ] Al abrir la ventana que va a cargar el firmware en el microcontrolador, oprimir el botón continuar, y oprimir el botón ok de la  ventana de dialogo.
\item[g. ] Una vez se inicia la carga del firmware en el microcontrolador aparece una ventana de color azul donde se muestra el estado de la carga del firmware\footnote{ Ver figura 11}, al finalizar se cerrara la ventana.
\item[h. ] Oprimir el botón salir.
\item[i. ] Conectar un usb-bluetooth a alguno de los puertos usb del computador y oprimir el botón ON de bluetooth que se encuentra en la parte izquierda del programa, dar ok a la ventana de dialogo.
\end{enumerate}

\begin{Figure}
\center
\includegraphics[width=8.cm, height=4cm]{fig/micro.png}
\captionof{figure}{Ventana al terminar de cargar correctamente el firmware en el microcontrolador atmega328P-PU.}
\label{fig:g11}
\end{Figure}
  
\subsection{Montaje de laboratorio}
\subsubsection{Materiales}
\begin{enumerate}
\item[a. ] Tres cuerdas de diferentes longitudes cuyo peso sea despreciable.
\item[b. ] Una masa que haga tensión sobre la cuerda.
\item[c. ] Una cinta métrica.
\item[d. ] Montaje eléctrico de free pops.
\item[e. ] Ordenador con Gnu-Linux instalado, conectado a una usb-bluetooth y el software de free pops.
\item[f. ] Un cronometro. 
\item[g. ] Un transportador. 
\end{enumerate}

\subsubsection{Montaje}

\begin{enumerate}
\item[a. ] Uno de los extremos de la cuerda, debe atarse a un eje de giro y el otro a la masa, formando de esta manera un péndulo,  debajo de la masa, debe colocarse el montaje eléctrico de free pops, asegurándose de que los diodos infrarrojos queden a menos de 12  centímetros  de la masa, figura 12.A y 12.B.
\item[b. ] El transportador, se  utiliza para medir el ángulo del que se suelta la masa, para que el péndulo comience a oscilar.
\end{enumerate}

\begin{Figure}	
\center
\begin{tabular}{|l|r|}
\hline
\includegraphics[width=4cm, height=4cm]{fig/m_1.png} & \includegraphics[width=4cm, height=4cm]{fig/m_2.png} \\ \hline
12.A & 12.B \\ \hline
\end{tabular}
\captionof{figure}{En la figura 12.A se muestra el montaje experimental de un péndulo y el montaje eléctrico de free pops debajo de este. La figura 12.B es una vista ampliada de la figura 12.A.}
\label{fig:g12}
\end{Figure}

\section{Experiencia de laboratorio}
Una vez hechos los pasos de la configuración experimental, se procede con el laboratorio como tal, para medir el periodo de oscilación del péndulo; para esto tenemos listo nuestro software free pops, con el bluetooth y todo ya preparado para funcionar; se saca la masa del péndulo de su reposo a un ángulo menor a 15 grados, y se suelta; paso seguido oprimimos el botón comenzar del software free pops, e inmediatamente el programa empezara a recibir datos del montaje, graficando en tiempo real, figura 13.
\\ \\ \\ \\
Una vez terminado de capturar los datos del sistema oscilante,  se debe medir la longitud de la cuerda que participo en la oscilación y también es necesario que una persona cuente las oscilaciones del péndulo, durante 30 segundos; otra persona le puede indicar cuando pasen 30 segundos, para que detenga su cuenta. de esta manera se puede comparar la frecuencia de oscilación  que arroja el software de free pops con la frecuencia de oscilación que tiene el péndulo.

\begin{Figure}
\center
\includegraphics[width=9.cm, height=4cm]{fig/graf1.png}
\captionof{figure}{Free pops captura  y gráfica los datos en  tiempo real al oprimir el botón comenzar.}
\label{fig:g13}
\end{Figure}

\section{Análisis de resultados}
Oprimir el botón analizar que se encuentra bajo el menú analizar datos, Lo que hará el programa, es ir al archivo donde se capturaron los datos y empezara hallar las pendientes, una vez obtenidas las pendientes lo que hará sera contar las oscilaciones que dio el péndulo,  y con las ecuaciones\cite{GIANCOLI} del movimiento armónico simple se obtiene  la frecuencia\ref{eqn:frecuencia}  de oscilación del péndulo\footnote{ Ecuación 1, f es la frecuencia, N es el numero de oscilaciones y t es el tiempo en el que dio las oscilaciones el péndulo.},ecuación 1; el periodo\ref{eqn:periodo} de oscilación\footnote{Ecuación 2, T es el periodo de oscilación, f es la frecuencia.}, ecuación 2; la frecuencia\ref{eqn:frecueniangular} angular\footnote{ Ecuación 3, $\omega^2$ es la frecuencia angular al cuadrado, $\pi $ es 3.141592, f es la frecuencia.}, ecuación 3; la longitud\ref{eqn:longitud} de la cuerda \footnote{ Ecuación 4, g es la aceleración gravitacional que tiene un valor de 9.81 $\frac{m}{s^2}$, $\omega^2$ es la frecuencia angular al cuadrado.}, ecuación 4; una vez termina de analizar los datos, aparecen los resultados en la ventana principal de free pops.

\begin{equation}
\label{eqn:frecuencia}
f = \frac{N}{t}
\end{equation}

\begin{equation}
\label{eqn:periodo}
T = \frac{1}{f}
\end{equation}

\begin{equation}
\label{eqn:frecueniangular}
\omega = 2 * \pi * f
\end{equation}

\begin{equation}
\label{eqn:longitud}
 l = \frac{g}{\omega^2}  
\end{equation}

\begin{Figure}
\center
\includegraphics[width=9.cm, height=4cm]{fig/analis1.png}
\captionof{figure}{Análisis de resultados del software free pops, con una masa de 50 gramos, la longitud de la cuerda es 34 cm, y el ángulo del que empieza a oscilar es de 10 grados;   el error en la medida del software es de 2.64\%; se obtuvo el mismo resultado con una masa de 20 gramos.}
\label{fig:g14}
\end{Figure}

\begin{Figure}
\center
\includegraphics[width=9.cm, height=4cm]{fig/analis2.png}
\captionof{figure}{Análisis de resultados del software free pops, con una masa de 50 gramos, la longitud de la cuerda es 39 cm, y el ángulo del que empieza a oscilar es de 10 grados;   el error en la medida del software es de 0,5\%; se obtuvo el mismo resultado con una masa de 20 gramos.}
\label{fig:g15}
\end{Figure}

Abajo del botón analizar, hay un botón que dice graficar, al oprimirlo, el programa muestra tres gráficas, fig. 16; en las cuales se aprecia los datos sin analizar, el análisis de pendientes, y un filtro de señal que se logra al obtener la frecuencia de oscilación y combinarla con sus  pendientes. 
\\ \\
El programa cuenta con un botón debajo del anterior que dice simulación, el software con los datos obtenidos y su respectivo análisis, realiza una simulación\cite{VISUAL}  del movimiento.

\begin{Figure}
\center
\includegraphics[width=9.cm, height=6cm]{fig/Graficas.png}
\captionof{figure}{Gráficas que arroja el programa free pops del análisis descrito en la figura 14.}
\label{fig:g16}
\end{Figure}

Por ultimo el programa cuenta con tres botones extras, uno que se encuentra arriba del botón firmware, que es la documentación de todo el programa y el montaje eléctrico, el botón detener que se encuentra debajo del botón comenzar, la función de este botón es el de dar la información de como detener la toma de datos del dispositivo  y el botón de limpiar que se encuentra abajo del botón detener, que reinicia el programa para comenzar con una nueva medición.


%\begin{Figure}
%\center
%\includegraphics[width=9.cm, height=4cm]{fig/Graficas1.png}
%\captionof{figure}{Gráficas que arroja el programa free pops del análisis descrito en la figura 15.}
%\label{fig:g17}
%\end{Figure}

\section{Conclusiones}
\begin{enumerate}
\item[a. ] El periodo de oscilación del péndulo no depende de la masa, sino de la longitud de la cuerda y la aceleración gravitacional del lugar donde se realice el movimiento armónico. 

\item[b. ] El software free pops, arroja un análisis de los resultados, con un margen de error muy pequeño, lo que da confianza al momento de llevar este modulo al aula de clase.

\item[c. ] Las practicas de laboratorio cuando se complementan con los laboratorios virtuales ofrecen una ventaja en el asimilar del saber especifico de algún tema particular de la ciencia; debido a la misma naturaleza didáctica demostrativa que esta conlleva.

\item[d. ] Las nuevas tecnologías de la información ofrecen un medio para lograr interactuar con los estudiantes y que estos muestren mejor disposición hacia la clase. 

\item[e. ] Las nuevas tecnologías de la información ofrecen una manera didáctica y en muchas ocasiones divertida para que los estudiantes analicen, saquen sus propias conclusiones y aprendan de experiencias hechas en el laboratorio, dando una ayuda importante a la clase magistral.

\item[f. ] El software y el hardware libre favorecen la educación y fomentan el aprendizaje de las ciencias.

\end{enumerate}
\end{multicols}
% The bibliography
\begin{thebibliography}{99}
\bibitem{REDISH} Redish, Edward F, \emph{Millikan Award Lecture 1998: Building as cience of teaching physics}, Am. J. Phys. 67 (1999).
\bibitem{GALILEO} Serna, E., José Marquiná, F., \& Fernández, E. GALILEO GALILEI. FACULTAD DE INGENIERÍAS, FUNDACION UNIVERSITARIA LUIS AMIGO.
\bibitem{VISUAL} García, C. P. (2008). Proyecto Physthones. Simulaciones Físicas en Visual Python (0.1). Un libro libre de Alqua.
\bibitem{GIANCOLI} Giancoli, D. C. (1984). General physics. Englewood Cliffs, NJ: Prentice-Hall.

\end{thebibliography}
\end{document}
