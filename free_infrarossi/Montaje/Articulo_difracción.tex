\documentclass[12]{article}
\usepackage[spanish,english]{babel}
%\usepackage[spanish]{babel}
\usepackage[utf8]{inputenc}
\usepackage{graphicx}
\usepackage{epsfig}
\usepackage{multirow}
\usepackage{multicol,caption}
\usepackage{amsthm} % Theorem Formatting
\usepackage{amssymb}    % Math symbols such as \mathbb
\usepackage{color}
\usepackage{hyperref}
\usepackage[none]{hyphenat}
\usepackage{appendix}
\renewcommand{\appendixname}{Anexo}
\renewcommand{\appendixtocname}{LISTA DE ANEXOS}
\renewcommand{\appendixpagename}{Anexos}
%\renewcommand{\tablename}{Tabla}
%\def\tablename{Cuadro}% por \def\tablename{Tabla}% 
\newenvironment{Figure}
{\par\medskip\noindent\minipage{\linewidth}}
{\endminipage\par\medskip}
\addto\captionsspanish{%
\def\tablename{Tabla}%
}
\topmargin  = 10pt
\oddsidemargin  = -0.5in
%\headheight = 12pt
%\headsep    = 15pt
%\footskip   = 15pt
\textheight = 21.5 cm
\textwidth  = 18.5cm
\tolerance=10000
\title{\bf{Calculo de la longitud de onda de la radiación infrarroja de un diodo led infrarrojo, utilizando el modulo motorizado infrarossi y su software de control Free infrarossi}}
\author{Julian Salamanca\footnote{jasalamanca@udistrital.edu.co}, Diego Parra\footnote{diegoestudianteud1@gmail.com} \\
  Universidad Distrital, Calle 3 No 26A-40 Bogotá-Colombia\\
  Grupo de Física e Informática ``FISINFOR''
}
\date{\today}
\begin{document}
%\def\tablename{Cuadro}% por \def \tablename{Tabla}% 
\renewcommand{\tablename}{Tabla}
\maketitle
\vspace{-0.8cm}
\selectlanguage{english}

\begin{abstract}
This work broken down in a clear construction of a motor vehicle, capable of sending and receiving radiation or light in the electromagnetic spectrum corresponding to the infrared wavelength, with calibrated sensors to illustrate the attenuation property, absorption and diffraction this type of radiation; this vehicle is controlled from a computer with Bluetooth and a GNU-Linux operating system, thanks to the serial communication module HC-05 and HC-06, the microcontroller Atmega 328P-PU and the computer; using free software and hardware.\\ \\
This work is expected to be welcomed by students and professionals who see in this an educational model with readily available materials and a large power measurement accuracy; it is also necessary to visualize the extent that free technologies have today, these technologies are the economic base of giant companies that benefit from it; and why not say, it has reached an age where information and means to transmit are available to anyone, such is the case we can take this technology and make it new tools to support the work of science education and learning in science; for this reason this world of free technologies are making, and combine performing with them new tools to continue this beautiful profession of teaching and learning that never ends.\\ 
{\bf{Keywords:}} Motor module, infrared sensors, microcontroller module bluetooth, electromagnetic wave.


\selectlanguage{spanish}
\begin{center}
{\bf{Resumen}} 
\end{center}

El presente trabajo desglosara de una manera clara la construcción de un vehículo motorizado, capaz de enviar y recibir radiación o luz en el espectro electromagnético correspondiente  a la longitud de onda infrarroja,   con sensores calibrados para ilustrar la propiedad de atenuación, absorción y  difracción de este tipo de radiación; este vehículo se controla  desde un computador con bluetooth y un sistema operativo GNU-Linux, gracias a la comunicación serial entre el modulo hc-05 o hc-06, el microcontrolador atmega  328P-PU y el ordenador; con ayuda de software y hardware libre. \\\\
Este trabajo espera ser acogido por estudiantes y profesionales que vean en este un modelo didáctico,  con materiales que consiguen fácilmente y  un gran poder de exactitud en las mediciones; además es necesario visualizar el alcance que las tecnologías libres tienen el día de hoy, estas tecnologías son la base  económica de empresas gigantescas que se benefician de ella; y porque no decirlo,  se ha alcanzado una era en donde la información y los medios que la transmiten están al alcance de cualquier persona,  tal es el caso que podemos tomar esta tecnología y hacer de ella nuevos instrumentos que respalden la labor de la enseñanza científica y el aprendizaje en ciencias; por esta razón se toma este mundo de tecnologías libres, y se combinan realizando con ellas nuevos instrumentos para continuar con este hermoso oficio de aprendizaje y enseñanza que no termina nunca. \\
{\bf{Descriptores:}} Modulo motorizado, sensores infrarrojos, microcontrolador, modulo bluetooth, ondas electromagnéticas. 
\end{abstract}

%tabla de contenido sin numeracion
%\renewcommand\contentsname{\centering TABLA DE CONTENIDO}
%\thispagestyle{empty}
%\setcounter{page}{1}
%\tableofcontents
%\clearpage

%lista de figuras
%\renewcommand\listfigurename{\centering LISTA DE FIGURAS}
%\listoffigures
%\clearpage

%lista de tablas
% \renewcommand\listtablename{\centering LISTA DE TABLAS}
% \listoftables
% \clearpage


\end{document}

